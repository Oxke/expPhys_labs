\input{../../preable_report.tex}

\title{Report Lab 1\\\small Experimental Physics for AI 2}
\author{G, C, D, O}
\date{First semester 2024 \-- 2025}

\begin{document}

\maketitle

\chapter{Measurement of the current-voltage characteristic of a resistor}
\chapter{Measurement of the current-voltage characteristic of a diode}
\section{Goal}
Now we want to measure the current-voltage characteristic of a diode, which
should not be linear. Indeed, according to Shockley's law, it is exponential:
\[
    I = I_0 \left( e^{\frac{qV}{gkT}} - 1 \right)
\]
where $I_0$ is the reverse saturation current, $q$ is the electron charge, \(k\)
is the Boltzmann constant, \(T\) is the temperature, and \(g\) is the diode
type-dependent constant. In this chapter we will try to verify this law.

Moreover for practical applications it's common practice to define the diode's
\emph{threshold voltage} as the voltage at which the diode starts conducting a
``significant'' current. We will try to measure this value as well.

\section{Method}
Using a similar setup as the one in part one, we recorded the measured values of
current at different voltages. The setup is shown in figure~\ref{fig:setup-diode}.

\begin{figure}[ht]
    \centering
    \incfig[.45]{setup-diode}
    \caption{Setup diode}\label{fig:setup-diode}
\end{figure}

\section{Data}
The data we collected is shown in table~\ref{tab:diode-data} and is represented
graphically in figure~\ref{tab:diode-graph}.

\begin{table}[ht]
    \centering
\begin{tabular}{ll|ll}
       \multicolumn{2}{l}{\bfseries Voltage (\(V\))} &
       \multicolumn{2}{l}{\bfseries Current (\(\mu A\))} \\
       \hline
       \csvreader[head to column names]{../data.csv}{}%
       {\Vdetapprox&\errVdet&\Idetapprox&\errIdet\\}
   \end{tabular}
       \caption{Data collected for the diode}\label{tab:diode-data}
   \end{table}
\end{document}

