\input{../../preable_report.tex}

\title{Report Lab 1\\\small Experimental Physics for AI 2}
\author{G, C, D, O}
\date{First semester 2024 \-- 2025}

\begin{document}

\maketitle

\chapter{Measurement of the current-voltage characteristic of a resistor}
\chapter{Measurement of the current-voltage characteristic of a diode}
\section{Goal}
Now we want to measure the current-voltage characteristic of a diode, which
should not be linear. Indeed, according to Shockley's law, it is exponential:
\[
    I = I_0 \left( e^{\frac{qV}{gkT}} - 1 \right)
\]
where $I_0$ is the reverse saturation current, $q$ is the electron charge, \(k\)
is the Boltzmann constant, \(T\) is the temperature, and \(g\) is the diode
type-dependent constant. In this chapter we will try to verify this law.

Moreover for practical applications it's common practice to define the diode's
\emph{threshold voltage} as the voltage at which the diode starts conducting a
``significant'' current. We will try to measure this value as well.

\section{Method}
Using a similar setup as the one in part one, we recorded the measured values of
current at different voltages. The setup is shown in
figure~\ref{fig:setup-diode}, where the voltmeter is a handheld Fluke
multimeter and the ammeter is a Agilent bench multimeter.

\begin{figure}[ht]
    \centering
    \incfig[.35]{setup-diode}
    \caption{Setup of the diode experiment: on the left the diagram showing
    the circuit made, on the right a photo of the setup}\label{fig:setup-diode}
\end{figure}

Later, in section~\ref{sec:analysis}, we will perform various fits to the data to
verify the exponential relation and estimate the values of the parameters.

\section{Data}\label{sec:data}
The data we collected is shown in table and represented graphically in
figure~\ref{fig:diode-data}. The bench multimeter for the current measurements
had an accuracy of \(\pm 0.05\% + 0.05\mu A\) in the \(500 \mu A\) range; and
the handheld multimeter had an accuracy of \(\pm 0.5\% + 0.002V\) in the \(2 V\)
range.

\begin{figure}[ht]
	{\small
\begin{tabular}[ht]{ll|ll}
       \multicolumn{2}{l}{\bfseries Voltage (\(V\))} &
       \multicolumn{2}{l}{\bfseries Current (\(\mu A\))} \\
       \hline
       \csvreader[head to column names]{../data.csv}{}%
       {\Vdetapprox&$\pm$ \errVdet&\Idetapprox&$\pm$ \errIdet\\}
\end{tabular}}
\begin{tikzpicture}[baseline=(current bounding box.base), scale=.9]
    \begin{semilogyaxis} [
        table/col sep=comma,
        title = {Diode current-voltage data},
        xlabel = {Voltage (\(V\))},
        ylabel = {Current (\(\mu A\))},
        % xmin = 2,
        no markers,
    ]

    \addplot+ [
            only marks, mark size=.2pt,
            error bars/.cd,
                x dir = both, x explicit,
                y dir = both, y explicit,
        ] table [x=Vdetapprox, y=Idetapprox, x error=errVdet, y error=errIdet] {../data.csv};
    
    \end{semilogyaxis}
\end{tikzpicture}
       \caption{Data collected for the diode}\label{fig:diode-data}
   \end{figure}

   \section{Analysis}\label{sec:analysis}

\end{document}

